\documentclass[a4paper,norsk,12pt]{article}
\usepackage{amsmath}
\usepackage[norsk]{babel}
\usepackage[utf8]{inputenc}
\usepackage{bm}


\title{\textbf{Obligatorisk innlevering 1, FYS1120-Elektromagnetisme}}
\author{Av: Laila Andersland}

\begin{document}
\maketitle

\textbf{Oppgave 1 \hspace{3mm}  Gradient, divergens og virvling} \\

\textbf{a)} \hspace{3mm}  { Finn gradienten} \\

$(i)$ \hspace{3mm} $f(x,y,z) = x^{2} y$ \\

\textbf{Svar:}

{Gradienten til skalarfeltet, \textit{f}, er et vektorfelt som peker i retningen til den} 

{største økningen av skalarfeltet. Gradienten finnes ved å bruke:}

\begin{equation}
\bigtriangledown f = ( \frac{\partial f}{\partial x} , \frac{\partial f}{\partial y} , \frac{\partial f}{\partial z} )
\end{equation}

slik at
$$
\bigtriangledown f = ( \frac{\partial x^{2}y}{\partial x} , \frac{\partial x^2 y}{\partial y} , \frac{\partial x^2 y}{\partial z} ) 
=
\def\doubleunderline#1{\underline{\underline{#1}}}
 (2xy , x^2 , 0) = \doubleunderline{(2xy,x^2)}
$$\\

$(ii)$ \hspace{3mm} $g(x,y,z) = x y z $ \

\textbf{Svar:}

Bruker (1) og gjør det samme:

$$
\def\doubleunderline#1{\underline{\underline{#1}}}
\bigtriangledown f  = \doubleunderline{(yz, xz, xy)}
$$ \\


$(iii)$ \hspace{3mm} $h(r,\theta ,\phi) = \frac{1}{r} e^{r^2} , r = \sqrt{x^2 + y^2 + z^2} $ \

\textbf{Svar:}

$$
\def\doubleunderline#1{\underline{\underline{#1}}}
\bigtriangledown h  = \doubleunderline{(- \frac{1}{r^2} e^{r^2} + \frac{1}{r} e^{r^2} ) \hat{e}_r}
$$ \\



(fyll inn utregningen her)\\

\textbf{b)} { Finn divergensen og virvlingen} \\

$(i)$ \hspace{3mm} $\textbf{u}(x,y,z) = (2xy, x^2, 0) $ \\

\textbf{Svar:}

Divergensen til vektorfeltene gir størrelsen til en kilde eller sluk i et gitt 

punkt i vektorfeltet, i form av en skalar. Divergensen finnes ved å regne 

ut:

$$
\bigtriangledown \cdot \textbf{u} = \frac{\partial}{\partial x} u_x + \frac{\partial}{\partial y} u_y + \frac{\partial}{\partial z} u_z \
$$ 

Setter så inn:

$$
\def\doubleunderline#1{\underline{\underline{#1}}}
\bigtriangledown \cdot \textbf{u} = \frac{\partial}{\partial x} 2xy + \frac{\partial}{\partial y} x^2 + \frac{\partial}{\partial z} 0 \
= 2y + 0 + 0 = \doubleunderline{ 2y} \
$$ \\

Virvlingen(curl)beskriver den infinitesimale rotasjonen av et vektorfelt, 

hvor lengden og retningen beskriver rotasjonen i et gitt punkt i feltet. 

Virvlingen finnes ved å regne ut:

$$
\bigtriangledown \times \textbf{u} = 
$$

$$
=\begin{vmatrix}
\hat i & \hat j & \hat k \\
\frac{\partial}{\partial x} & \frac{\partial}{\partial y} & \frac{\partial}{\partial z} \\
u_x & u_y & u_z
\end{vmatrix}
=\begin{vmatrix}
\hat i & \hat j & \hat k \\
\frac{\partial}{\partial x} & \frac{\partial}{\partial y} & \frac{\partial}{\partial z} \\
2xy & x^2 & 0
\end{vmatrix}
$$

$$
= \left( \frac{\partial}{\partial y} (0) - \frac{\partial}{\partial z} (x^2) \right) \hat i + \left(\frac{\partial}{\partial z} (2xy) - \frac{\partial}{\partial x} (0) \right) \hat j + \left(\frac{\partial}{\partial x} (x^2) - \frac{\partial}{\partial y} (2xy) \right) \hat j
$$

$$
\def\doubleunderline#1{\underline{\underline{#1}}}
\left(0 - 0 \right) \hat i + \left(0 - 0 \right) \hat j + \left(2x - 2x \right) \hat k = \doubleunderline{\vec{0}}
$$\\



$(ii)$ \hspace{3mm} $\textbf{v}(x,y,z) = (e^{yz}, ln(xy), z) $ \\

\textbf{Svar:}

Divergensen er:


$$
\bigtriangledown \cdot \textbf{v} = \frac{\partial}{\partial x} v_x + \frac{\partial}{\partial y} v_y + \frac{\partial}{\partial z} v_z \  = \frac{\partial}{\partial x} (e^{yz})+ \frac{\partial}{\partial y} ln(xy) + \frac{\partial}{\partial z} z \
$$ 

$$
\def\doubleunderline#1{\underline{\underline{#1}}}
= 0 + \frac{1}{y} + 1 =  \frac{1}{y} + 1 = \doubleunderline{ \frac{1}{y} (y+1)} \
$$ \\

Og virvlingen blir:

$$
\bigtriangledown \times \textbf{v} 
= \left( \frac{\partial}{\partial y} (z) - \frac{\partial}{\partial z} \Big(ln(xy)\Big) \right) \hat i + \left(\frac{\partial}{\partial z} (e^{yz}) - \frac{\partial}{\partial x} (z) \right) \hat j + \left(\frac{\partial}{\partial x} \Big(ln(xy)\Big) - \frac{\partial}{\partial y} (e^{yz}) \right) \hat k
$$

$$
=\left(0 - 0 \right) \hat i + \left(ye^{yz} - 0 \right) \hat j + \left(\frac{1}{xy} - ze^{yz} \right) \hat k 
$$

$$
\def\doubleunderline#1{\underline{\underline{#1}}}
= 0 \hat i + ye^{yz} \hat j + \left( \frac{1}{xy} - ze^{yz} \right)\hat{k} = \doubleunderline{ ye^{yz} \hat j + \left( xy^{-1} - ze^{yz} \right)\hat{k} }
$$\\

$(iii)$ \hspace{3mm} $\textbf{w}(x,y,z) = (yz, xz, xy) $ \\

\textbf{Svar:}
Divergensen:\

$$
\bigtriangledown \cdot \textbf{w} = \frac{\partial}{\partial x} w_x + \frac{\partial}{\partial y} w_y + \frac{\partial}{\partial z} w_z \  = \frac{\partial}{\partial x} (yz)+ \frac{\partial}{\partial y} (xz) + \frac{\partial}{\partial z} (xy) \
$$ 

$$
\def\doubleunderline#1{\underline{\underline{#1}}}
= 0 + 0 + 0 =   \doubleunderline{ 0} \
$$ \

Og virvlingen blir:

$$
\bigtriangledown \times \textbf{w} 
= \left( \frac{\partial}{\partial y} (xy) - \frac{\partial}{\partial z} (xz) \right) \hat i + \left(\frac{\partial}{\partial z} (yz) - \frac{\partial}{\partial x} (xy) \right) \hat j + \left(\frac{\partial}{\partial x} xz - \frac{\partial}{\partial y} (yz) \right) \hat k
$$

$$
=\left(x - x \right) \hat i + \left(y - y \right) \hat j + \left({z} - z \right) \hat k 
$$

$$
\def\doubleunderline#1{\underline{\underline{#1}}}
= 0 \hat i + 0 \hat j + 0\hat{k} = \doubleunderline{ \vec{0} } 
$$



$(iv)$ \hspace{3mm} $\textbf{a}(x,y,z) = (y^2 z, -z^2 sin{y} + 2xyz, 2z cos{y} + y^2 x) $ \\

\textbf{Svar:}

Divergensen:\

$$
\bigtriangledown \cdot \textbf{a} = \frac{\partial}{\partial x} a_x + \frac{\partial}{\partial y} a_y + \frac{\partial}{\partial z} a_z \  = \frac{\partial}{\partial x} (y^2z)+ \frac{\partial}{\partial y} (-z^2 siny + 2xyz) + \frac{\partial}{\partial z} (2zcosy + y^2x) \
$$ 

$$
\def\doubleunderline#1{\underline{\underline{#1}}}
= 0 - z^2cosy + 2xz + 2cosy =   \doubleunderline{ 2xz + cosy(2-z^2)} \
$$ \

Og virvlingen blir:

$$
\bigtriangledown \times \textbf{a} 
= \left( \frac{\partial}{\partial y} (2zcosy + y^2x) - \frac{\partial}{\partial z} (-z^2 siny + 2xyz) \right) \hat i 
$$
$$
+ \left(\frac{\partial}{\partial z} (y^2z) - \frac{\partial}{\partial x} (2zcosy + y^2x) \right) \hat j + \left(\frac{\partial}{\partial x} (-z^2 siny + 2xyz) - \frac{\partial}{\partial y} (y^2z) \right) \hat k
$$

$$
=\left((-2zsiny + 2yx) - (-2zsiny+2xy) \right) \hat i + \left(y^2 - y^2 \right) \hat j + \left(2yz -2yz \right) \hat k 
$$

$$
\def\doubleunderline#1{\underline{\underline{#1}}}
= 0 \hat i + 0 \hat j + 0\hat{k} = \doubleunderline{ \vec{0} } 
$$

\textbf{c)} \\

\textbf{Svar:}

At et felt er \textit{konservativt}, betyr at feltet er virvlingsfritt, altså at 


\begin{equation}
\bigtriangledown \times \textbf{a} = \vec{0}
\end{equation}

Dette betyr at feltet kan representeres som gradienten av et potensialfelt:

\begin{equation}
\textbf{a} = - \bigtriangledown V 
\end{equation}

hvor V kalles potensialet. \\


At et gravitasjonsfelt er konservativt betyr at integralet av en kurve i 

dette feltet, kun er avhengig av endepunktene. Vi kan da si at

gravitasjonsfeltet, \textbf{g}, er vei-uavhengig.\\

I oppgave \textbf{b)} er \textbf{u}, \textbf{w} og \textbf{a} virvlingsfrie og dermed konservative felt, 

hvor da f er potensialet til \textbf{u}, g er potensialet til \textbf{w} og h er potensialet 

til \textbf{a}.\\


\textbf{d)} \hspace{3mm}  Bruk laplaceoperatoren på disse skalarfeltene:\\

$(i) j(x,y,z) = x^2 + xy + yz^2$ 

\textbf{Svar:}\\
$$
\bigtriangledown^2 j = \bigtriangledown \cdot (\bigtriangledown j )
$$

$$
(\bigtriangledown \cdot j) = (2x + y, x + z^2, 2yz)
$$

$$
\def\doubleunderline#1{\underline{\underline{#1}}}
\doubleunderline{\bigtriangledown ( \bigtriangledown \cdot j) }= 2 + 0 + 2 =  \doubleunderline{4}
$$\\


$(ii)$ \hspace{3mm} $h(r,\theta ,\phi) = \frac{1}{r} e^{r^2}  $ \

\textbf{Svar:}
Fra oppgave \textbf{a} har vi at:

$$
\def\doubleunderline#1{\underline{\underline{#1}}}
\bigtriangledown h  = \doubleunderline{(- \frac{1}{r^2} e^{r^2} + \frac{1}{r} e^{r^2} ) \hat e_r}
$$ \\

$$
\bigtriangledown \cdot \bigtriangledown h = \frac{1}{r^2} \frac{\partial}{\partial r} (r^2 (- \frac{e^{r^2}}{r^2} + 2 e^{r^2}))
$$

$$
= \frac{1}{r^2} \frac{\partial}{\partial r} (-e^{r^2} +2 r^2 e^2) = \frac{1}{r^2} (-2re^{r^2} + 4 r e^{r^2} + 2r^2 2re^{r^2})
$$

$$
=\frac{1}{r^2} (2re^{r^2} + 4r^3 e^{r^2}) = \frac{2 e^{r^2}}{r} + 4re^{r^2}
$$

$$
\def\doubleunderline#1{\underline{\underline{#1}}}
=\doubleunderline{(\frac{1}{r} + 2r)2e^{r^2}}
$$\\


\textbf{Oppgave 2} \hspace{3mm}  Vektoridentiteter \\

vis følgende identitet:

$ \textbf{a} \times (\textbf{b} \times \textbf{c} ) = \textbf{b}(\textbf{a} \cdot \textbf{c} - \textbf{c}(\textbf{a} \times \textbf{b}9 $\\


\textbf{Oppgave 3} \hspace{3mm} Fluks og Gauss' teori

\textbf{a)} 

$$ \textbf{v}(x,y,z) = (2x - y) \hat i y^2 \hat j - y^2 z \hat k $$\\


\textbf{Oppgave 4} \hspace{3mm} Linjeintegral og Stokes' teorem\\

$$ \textbf{w} (x,y,z) = (2x - y) \hat i -y^2 \hat j - y^2 z \hat k $$
\\
\textbf{a)} Regn ut divergensen til \textbf{w}
\\

\textbf{Svar:}

$$
\nabla \cdot \textbf{w} = \frac{\partial}{\partial x} (2x-y) + \frac{\partial}{\partial y} (-y^2)+ \frac{\partial}{\partial z} (-y^2 z) \ 
$$

$$
\def\doubleunderline#1{\underline{\underline{#1}}}
= \doubleunderline{2 - 2y - y^2}
$$

\textbf{b)} Regn ut virvlingen til \textbf{w}


$$
\nabla \times \textbf{w}
=\begin{vmatrix}
\hat i & \hat j & \hat k \\
\frac{\partial}{\partial x} & \frac{\partial}{\partial y} & \frac{\partial}{\partial z} \\
(2x-y) & (-y^2) & (-y^2 k)
\end{vmatrix}
= (-2 y k - 0) \hat i + (0 - 0) \hat j + (0+1) \hat k
$$

$$
\def\doubleunderline#1{\underline{\underline{#1}}}
= \doubleunderline{-2yz \hat i + \hat k}
$$ \\

\textbf{c)} Parametrisér kurven {$\Gamma$} som en vektor \bm{$\Gamma$} (\textit{t}), og finn d \bm{$\Gamma$} uttrykt 

\hspace{4mm} ved d{\textit{t}}\\

\textbf{Svar:} \\

$$
x = cos(t)
$$
$$
y = sin(t)
$$
$$
z = 1
$$

$$\bm{\Gamma} (\bm{x})= x \hat i + y \hat j + \hat k $$

$$\bm{\Gamma} ({t}) = cos t \hat i + sin t \hat j + \hat k $$

$$ \dfrac{\bm{\Gamma} ({t})}{dt} = -sin t \hat i + cos t \hat j $$


$$
\def\doubleunderline#1{\underline{\underline{#1}}}
\doubleunderline{d \bm{\Gamma}   = -sin t \hat i + cos t \hat j dt}
$$\\

\textbf{d)} Regn ut sirkulasjonen \textit{C}, til \textbf{w} rundt $\Gamma$.\\

$$
\textbf{w} (\bm{\Gamma}) =  (2 cos t - sin t) \hat i - sin^2 t \hat j - sin^2 t \hat k
$$

Sirkulasjonen er 

$$ C = \oint \textbf{w} \cdot d {\bm{\Gamma} } $$

Begynner først med ($\bm{w} \cdot d {\bm{\Gamma} }$):

$$
\bm{w} \cdot d {\bm{\Gamma} } = \Big( (2 cos t - sin t) ) \hat i - sin^2 t \hat j - sin^2 t \hat k \Big) \Big( -sin t \hat i + cos t \hat j + 0 \hat k \Big) dt
$$

$$
= (-2cos t sin t + sin^2 t cos t + sin^2 t ) dt 
$$

$$
C = \int_0^{2 0pi}  (-2cos t sin t + sin^2 t cos t + sin^2 t ) dt
$$

Deler opp og regner først ut

$$
-2 \int (cos t sin t ) dt + \int (sin^2 cos t) dt + \int sin^2 t dt
$$


$$\Big[cos^2 t + \dfrac{1}{3} sin^2 t + \dfrac{1}{2} (t - sin t cos t) \Big]_0^{2 \pi} = [cos^2 2\pi + \dfrac{1}{3} sin^3 2\pi + \dfrac{1}{2} ({2 \pi} - sin {2 \pi} cos {2 \pi})] 
$$

$$
-[cos^2 0 + \dfrac{1}{3} sin^3 0 + \dfrac{1}{2} ({0} - sin {0} cos {0})] 
$$

$$ = [1+0 + \pi] - [1 + 0 + 0] = \pi $$\\


\def\doubleunderline#1{\underline{\underline{#1}}} 
Sirkulasjonen er dermed \doubleunderline{C = $\pi$}
\\

\textbf{e)} Bruk Stokes' teorem til å finne sirkulasjonen:\\

\textbf{Svar:}

Stokes' teorem:

$$
\int_\gamma \textbf{w} \cdot d \Gamma = \int_A \nabla \times \textbf{w} \cdot d \sigma
$$


Har virvlingen fra \textbf{b}):

$$\nabla \times \textbf{w} = {-2yz \hat i + \hat k}$$

Og at enhetsnormalvektoren går i positiv $\hat k $ retning.

Har da at:

$$
C = \int_0^1 \int_0^{2 \pi} \nabla \times \textbf{w} \times \textbf{n} r d \theta dr = \int_0^{2 \pi} \int_0^1 r dr d \theta
$$

$$
2 \pi \int_0^1 r dr = \dfrac{1}{2} 2 \pi = \doubleunderline{\pi}
$$

____________________________________________

\end{document}
